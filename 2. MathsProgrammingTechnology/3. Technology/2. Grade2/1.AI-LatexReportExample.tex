% --- LaTeX is used for professional, highly structured documents. ---
% The document class defines the fundamental layout (e.g., article, report, book).
\documentclass[11pt, a4paper]{article}

% --- UNIVERSAL PREAMBLE BLOCK ---
% Set geometry for clean margins
\usepackage[a4paper, top=2.5cm, bottom=2.5cm, left=2cm, right=2cm]{geometry}
\usepackage{fontspec}

% Set up language support
\usepackage[english, bidi=basic, provide=*]{babel}

\babelprovide[import, onchar=ids fonts]{english}

% Set default/Latin font to Sans Serif (clean and modern)
\babelfont{rm}{Noto Sans}

% Required packages for math and tables
\usepackage{amsmath}
\usepackage{booktabs}
\usepackage{hyperref} % Must be the last package loaded

% Define document information
\title{Introduction to Document Typesetting}
\author{A Digital Learner}
\date{\today}
% ---------------------------------

\begin{document}

% The \maketitle command generates the title block
\maketitle

\begin{abstract}
This brief document serves as a minimal example of a LaTeX file. It demonstrates basic structural elements, including sections, simple text formatting, and a mathematical equation, which is where LaTeX truly excels.
\end{abstract}

\section{Document Structure}
LaTeX automatically manages all numbering for sections, subsections, and figures. This makes maintaining a consistent and clean table of contents very easy for long reports.
``
\subsection{Formatting Basics}
To emphasize text, we use commands like \textbf{bold text} and \textit{italic text}. Notice the control over the paragraph layout and spacing.
%
\section{Mathematics Example}
Complex formulas are easily rendered with professional quality. The equation below is an example of an integral.
$$
\int_{0}^{\pi} \sin(x) \, dx = [-\cos(x)]_{0}^{\pi} = 1 - (-1) = 2
$$
This level of precision is why LaTeX is the standard for scientific papers.

\section{Data Presentation}
For data, we use the \texttt{tabular} environment, often enhanced by the \texttt{booktabs} package for cleaner lines.

\begin{table}[htbp]
    \centering
    \caption{Simple Data Table}
    \label{tab:data}
    \begin{tabular}{lcc}
        \toprule
        Category & Value 1 & Value 2 \\
        \midrule
        A & 10 & 25 \\
        B & 15 & 30 \\
        \bottomrule
    \end{tabular}
\end{table}

\end{document}